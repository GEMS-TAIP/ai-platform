\documentclass{article}

\usepackage{geometry}

% Make header with name and date etc.
\usepackage{fancyhdr}
\lhead{Your Name\\BIOS 29326}
\rhead{\today\\CT Problem Set}
\thispagestyle{fancy}

\usepackage[utf8]{inputenc}
\setlength{\parindent}{0pt} % Don't indent new paragraphs
\setlength{\headheight}{24pt} 

\begin{document}

\section{Radon Transform of a Gaussian}

Calculate the Radon transform of $p(\xi, \phi)$ of $f(x,y) = e^{-x^2 - y^2}$. (Hint: there is symmetry you can exploit to simplify this problem).

\vspace{12pt}

% Your Answer goes here

\section{Radon Transform of Shifted Function}

Show that if the Radon transform of $f(x,y)$ is $p(\xi,\phi)$, then the Radon transform of $f(x-x_0,y-y_0)$ is $p(\xi-x_0\cos{\phi} - y_0\sin{\phi})$. Also give a graphical explanation of this result. [Hint: this is somewhat easier to prove if you use the delta function form of the Radon transform given in the book: $p(\xi,\phi) = \int_{-\infty}^\infty \int_{-\infty}^{\infty} f(x,y)\delta(x\cos{\phi} + y\sin{\phi} - \xi)\,dxdy$.]

\vspace{12pt}

% Your Answer goes here

\section{Radon transform consistency conditions}

Let $p(\xi,\phi)$ be a parallel-beam sinogram and $P_\phi(\nu)$ be its 1D Fourier transform with respect to $\xi$ for fixed $\phi$, as defined in the lecture. Show that 

$$
P_{\phi + \pi}(\nu) = P_{\phi}(-\nu)
$$

\vspace{12pt}

% Your Answer goes here

\section{Problem 6.12 from Prince book}

Consider an object comprising two small metal pellets located at $(x,y) = (2,0)$ and $(2,2)$ and a pice of wire stretched straight between $(0,-2)$ and $(0,0)$.\\

(a) Sketch this object. Assume $N$ photons are fired at each lateral position $\ell$ in a parallel-ray configuration. For simplicity, assumes that each metal object stops 1/2 the photons that are incident upon it no matter what angle it is hit.\\

(b) Sketch the number of photons you would expect to see as a function of $\ell$ for $\theta = 0^\circ$ and $\theta = 90^\circ$. \\

(c) Draw the projections you would see at $\theta = 0^\circ$ and $\theta = 90^\circ$.\\

(d) Sketh the backprojection image you would get at $\theta = 0^\circ$ (without filtering). 

\vspace{12pt}

% Your Answer goes here

\end{document}
